%% LyX 1.6.5 created this file.  For more info, see http://www.lyx.org/.
%% Do not edit unless you really know what you are doing.
\documentclass[UTF8]{ctexart}
\usepackage[T1]{fontenc}
\usepackage{CJKutf8}
\usepackage{color}


\usepackage{url}
\usepackage[unicode=true, pdfusetitle,
 bookmarks=true,bookmarksnumbered=true,bookmarksopen=true,bookmarksopenlevel=2,
 breaklinks=true,pdfborder={0 0 1},backref=false,colorlinks=true]
 {hyperref}
\begin{document}
\begin{CJK}{UTF8}{}%

\title{统计之都GIT使用说明}


\author{肖楠%
\thanks{Email:road2stat@gmail.com;个人主页:http://www.road2stat.com%
} \and 谢益辉%
\thanks{Email:xie@yihui.name;个人主页:http://cos.name%
}}

\maketitle
本文描述了如何使用GIT在统计之都(COS)上合作编写文档,期待众多会员能将自己所学以及在COS的收获分享出来,形成有质量、有水平的小论文或书籍、手册。


\section{GIT入门}

这份文档(英文)描述了在Windows下从安装到使用GIT的过程:
\begin{itemize}
\item \url{http://nathanj.github.com/gitguide/tour.html}
\end{itemize}
如果想看中文文档,下面这篇PDF文档详细描述了GIT的工作流程:
\begin{itemize}
\item \url{http://linuxgem.org/user_files/linuxgem/Image/git-tutor.pdf}
\end{itemize}

\section{GitHub}

GitHub是一个提供Git服务的网站(\url{http://github.com}),注册会员可免费获得300M公开空间,若需要私人空间或者更多服务,则需要额外付费。
\begin{itemize}
\item 注册
\item 创建密钥
\item 创建本地仓库,向服务器提交更改
\item 获取他人的工程(通过Fork),如何合并二人的工作?
\end{itemize}
本文便是基于GitHub协作的一个例子:\url{http://github.com/yihui/COS}


\section{COS与GIT和\protect\LaTeX{}}

COS服务器上提供了GIT和\LaTeX{}环境,工作流程为:
\begin{itemize}
\item 定期从GitHub取回工程(git clone?)
\item 编译tex文件为PDF(也许会结合R/Sweave)
\item 自动更新工程页面(动态写HTML)
\end{itemize}
具体的设置待定。
\end{CJK}

\end{document}
